\documentclass[10pt,a4paper]{jsarticle}

\usepackage{amsmath,amssymb}
\usepackage[dvipdfmx]{graphicx}
\usepackage{float}
\usepackage{tikz}
\usepackage{circuitikz}
\usepackage{siunitx}
\usepackage{at}
\usepackage{comment}
\usepackage[hang,small,bf]{caption}
\usepackage[subrefformat=parens]{subcaption}

\captionsetup{compatibility=false}

\usetikzlibrary{intersections,calc,arrows.meta}

\numberwithin{equation}{section}
\numberwithin{figure}{section}
\numberwithin{table}{section}

\begin{document}
\section{実験の目的}
今回の実験は、以下のことを目的とする。
  \begin{itemize}
    \item [(1)] コイル($L$)と抵抗($R$)を直列接続した交流回路の電源電圧及び抵抗両端の電圧を測定し、位相角(位相差)等を求め、電圧等がベクトルであることを理解する。
    \item [(2)] 回路中の1つの値(例えば$R$)が変化するとベクトルの大きさと方向が変化する。ベクトルの先が変わる様子を連続的に書き、交流回路の特性を学ぶ。
  \end{itemize}

\section{原理}
  図\ref{fig:RL直列交流回路}の交流回路において、電源電圧の大きさ(実効値、以下同じ)を$V$、抵抗$R$およびコイル$L$の端子電圧の大きさを$V_R$、$V_L$、回路電流の大きさを$I$とするとき、以下の関係式が成立する。ここで、ベクトル量には量記号の上に($\rightarrow$)をつける。$\omega$は角周波数、$f$は周波数である。GNDは、電位の基準(零電位)で、電圧を測る基準である。

  \begin{equation}
    \begin{split}
        &\vec{V}=\vec{V_R} + \vec{V_L}
    \end{split}
  \end{equation}
  \begin{equation}
    \begin{split}
        &\vec{V_R}=R\vec{I},  \quad\quad V_R=RI
    \end{split}
  \end{equation}
  \begin{equation}
    \begin{split}
        &V=\sqrt{V^2_R + V^2_L}=\sqrt{R^2 + \left( \omega L \right)^2}, \omega=2 \pi f
    \end{split}
  \end{equation}
  \begin{equation}
    \begin{split}
        &V^2=V^2_R + V^2_L
    \end{split}
  \end{equation}
  \begin{equation}
    \begin{split}
        &\theta= \tan^{-1}\dfrac{V_L}{V_R}=\cos^{-1}\dfrac{V_R}{V} 
    \end{split}
  \end{equation}
  
  %図2.1: RL直列交流回路(実験回路接続図)
  \begin{figure}[H]
    \begin{center}
      \begin{circuitikz}
        \draw(0,0)
          to [vsourcesin] (0,6)
          to [short](2,6);
        \draw(5,0)
          to [short] (2,0)
          to[short] (0,0);
        \draw (2,6)
          to[short] (5,6)
          to[L](5,3)
          to[european resistor](5,0)
          to [short] (3.5,0);
        \draw (5,3)
          to[short] (7,3)
          to[short] (7,2);
        \draw (7,1.5) circle (0.5);
        \draw (7,1)
          to[short] (7,0)
          to[short] (5,0);
        \draw (5,6)
          to[short] (9,6)
          to[short] (9,2);
        \draw(9,1.5) circle (0.5);
        \draw(9,1)
          to[short] (9,0)
          to[short](7,0);
        \draw (0,0) to (0,-0.1) node[cground]{};
        \draw (5.7,4.8) node[anchor=north]{$L$};
        \draw (5.7,1.8) node[anchor=north]{$R$};
        \draw (2.5,6) node[anchor=north]{I};
        \draw (10,1) node[anchor=north]{OSC.CH1};
        \draw (8,1) node[anchor=north]{OSC.CH2};
        \draw [->,>=stealth,very thick] (2,6)--(3.5,6);
        \fill [black] (0,0) circle (0.06);
        \fill [black] (5,0) circle (0.06);
        \fill [black] (7,0) circle (0.06);
        \fill [black] (5,6) circle (0.06);
        \fill [black] (5,3) circle (0.06);
        \draw (6.7,1.3)--(7.1,1.7)--(7.1,1.4)--(7.35,1.6);
        \draw (8.7,1.3)--(9.1,1.7)--(9.1,1.4)--(9.35,1.6);

      \end{circuitikz}
     \end{center}
     \caption{RL直列交流回路(実験回路接続図)}\label{fig:RL直列交流回路}
  \end{figure}

  電流$I$を基準に描いたベクトル図を図2.2(a)に、電圧$V$を基準に書いたベクトルの三角形を図2.2(b)に示す。電流$I$は電圧$V$に対し位相が$\theta$だけ進んでいる(図2.2(a))。電流$I$と電圧$V_R$は同相(同方向)である。
  %図12.2: (a) 電流基準ベクトル図、(b)電圧基準ベクトルの三角形
  \begin{figure}[H]
      \centering
        \begin{circuitikz}
          %a
          \draw [->,>=stealth,very thick](0,0)--(0,3);
          \draw [->,>=stealth,very thick](0,0)--(3,0);
          \draw [->,>=stealth,very thick](0,0)--(3,3);
          \draw [dashed,semithick](0,3)--(3,3);
          \draw [dashed,semithick](3,0)--(3,3);
          \draw [->,>=stealth,thick](3,0)--(4,0);
          \draw (-0.3,2) node[anchor=north]{$\vec{V_L}$};
          \draw (1.6,0.6) node[anchor=north]{$\vec{V_R}$};
          \draw (3.4,0.6) node[anchor=north]{$\vec{I}$};
          \draw (1.3,3) node[anchor=north]{$\vec{V}$};
          \draw (1.2,2.5) node[anchor=north]{大きさ$V$};
          \draw (0.5,0) arc (0:45:0.5);
          \draw (0.6,0.5) node[anchor=north]{$\theta$};
          \draw (-1,5) node[anchor=north]{(a)};

          %b
          \draw [->,>=stealth,very thick] (6,2.7)--(9,0);
          \draw [->,>=stealth,very thick] (6,2.7)--(11,2.7);
          \draw [<-,>=stealth,very thick] (11,2.7)--(9,0);
          \draw (10.5,1.4) node[anchor=north]{$\vec{V_L}$};
          \draw (7,1.4) node[anchor=north]{$\vec{V_R}$};
          \draw (8.3,3.4) node[anchor=north]{$\vec{V}$};
          \draw (6.5,2.7) arc (0:-45:0.5);
          \draw (6.7,2.6) node[anchor=north]{$\theta$};
          \draw (6.5,5) node[anchor=north]{(b)};

        \end{circuitikz}
    \caption[margin=100pt]{(a) 電流基準ベクトル図、(b)電圧基準ベクトルの三角形}\label{fig:原理のベクトル2}
  \end{figure}

  電源電圧の大きさ$V$およびコイルのインダクタンス$L$を一定にして、抵抗$R$の値を変化させると、抵抗$R_1$,$R_2$,$R_3$$\cdot\cdot\cdot$に対し、ベクトル$\vec{V_R}$は図2.3(a)のように変化する。図2.3(a)に示す各点$P_1$,$P_2$,$P_3$$\cdot\cdot\cdot$は抵抗値の変化に対応して移動する。各点$P_1$,$P_2$,$P_3$$\cdot\cdot\cdot$の軌跡は半円になる。半円の直径は$V$である。このとき、電流ベクトル$\vec{I}$のベクトル軌跡も、図2.3(b)のような半円になる。半円の直径は$\dfrac{V}{\omega L}$である。点$P_0$は$R=0$のときである。このとき、位相角$\theta=90^\circ$となる。
  \begin{figure}[H]
    \begin{center}
      \begin{circuitikz}
        %a
        \draw[->,>=stealth, very thick] (0,0)--(4,0);
        \draw [dashed,semithick] (0,0) arc (-180:0:2);
        \draw[->,>=stealth, very thick] (0,0)--(0.3,-1);%P1
        \draw[->,>=stealth, very thick] (0,0)--(1,-1.7);%P2
        \draw[->,>=stealth, very thick] (0,0)--(2,-2);%P3
        \draw (0.3,-1)--(4,0);
        \draw (1,-1.7)--(4,0);
        \draw (2,-2)--(4,0);
        \draw [->,>=stealth,thin] (-0.3,-0.6)--(0.18,-0.7);
        \draw [->,>=stealth,thin] (0.18,-1.4)--(0.8,-1.5);
        \draw [->,>=stealth,thin] (1,2.-4)--(1.7,-1.7);
        \draw [->,>=stealth,semithick] (3,-2.2) arc (-70:-40:2.3);
        \draw (0,1.5) node[anchor=north]{(a)};
        %capital-set
        \scalebox{0.7}{
        \draw (-0.9,-0.9) node{$\vec{V_{R1}}$};
        \draw (-0.3,-2.2) node{$\vec{V_{R2}}$};
        \draw (1.1,-3.2) node{$\vec{V_{R3}}$};
        \draw (0,-1.4) node{$P_{1}$};
        \draw (1,-2.6) node{$P_{2}$};
        \draw (2.8,-3.2) node{$P_{3}$};
        \draw (5,-3.5) node{R増加};
        \draw (2.8,0.5) node{$\vec{V}$};
        }

        %b
        \draw (6,1.5) node[anchor=north]{(b)};
        \draw [->,>=stealth,very thick] (7,1)--(7,-3);
        \draw [->,>=stealth,very thick] (7,1)--(9,1);
        \draw [<->,>=stealth,semithick] (6.7,1)--(6.7,-3);
        \draw [dashed, semithick] (7,1) arc (90:-90:2);
        \draw [->,>=stealth,very thick] (7,1)--(9,-1);
        \draw [->,>=stealth,very thick] (7,1)--(8.8,-1.9);
        \draw [->,>=stealth,very thick] (7,1)--(8.4,-2.6);
        \scalebox{0.6}{
          \draw (13,-2.7) node{$\vec{I_1}$};
          \draw (13.7,-2.3) node{$\vec{I_2}$};
          \draw (14.4,-1.8) node{$\vec{I_3}$};
          \draw (11.7,-5.3) node{$P_0$};
          \draw (14.2,-4.7) node{$P_1$};
          \draw (15.2,-3.3) node{$P_2$};
          \draw (15.5,-1.6) node{$P_3$};
          \draw (10.4,-1.6) node{$\dfrac{V}{\omega L}$};
          \draw (14.4,1.3) node{$\vec{V}$};
          }

      \end{circuitikz}
    \end{center}
    \caption{(a)$V_R$のベクトル軌跡($R$を変化)、(b)$I$のベクトル軌跡($R$を変化)}\label{fig:原理のベクトル3}
  \end{figure}

\newpage
\section{実験}
  今回の実験では、図\ref{fig:RL直列交流回路}の回路において2つの実験(実験1と実験2)を行い、測定した。
\subsection{実験方法}
  \subsubsection{実験1}
    \begin{itemize}
      \item [(1)]図\ref{fig:RL直列交流回路}のように配置、配線をした。(この時、アース側の配線はすべて黒色にした。)
      \item [(2)]オシロスコープの「Measure」機能を使い、周波数$f$とCH1の周期RMS$V_{rms1}$とCH2の周期RMS$V_{rms2}$を表示させた。
      \item [(3)]周波数$f$を$\SI{100}{\kilo\hertz}$になるように関数発生器を調整した。
      \item [(4)]次に、可変抵抗器を$\SI{10}{\ohm}$に設定したのちに、オシロスコープの画面のCH1の周期RMS$V_{rms1}$が$\SI{2.0}{\volt}$になるように、関数発生器の「AMPLITUDE」を調整した。この時の、CH2の周期RMS$V_{rms2}$を表に記録した。
      \item [(5)]以降、表\ref{tab:実験1の表の例}のように、左の抵抗$R$の値から順々に測定した。この時、抵抗$R$を変えた際に、オシロスコープの画面の周期RMS$V_{rms1}$が$\SI{2.0}{\volt}$になっていない場合は、(4)の時と同じように調整した。
    \end{itemize}
    \begin{table}[H]
      \caption{設定する抵抗値と対応する電力ラベル}\label{tab:実験1の表の例}
        \centering
        \scalebox{0.8}{
          \begin{tabular}{|c||c|c|c|c|c|c|c|c|c|c|c|c|c|c|c|c|c|c|}
            \hline
              $R$ $\si{\ohm}$&10&16&25&40&63&100&160&250&400&630&1000&1600&2500&4000&6300&10000 \\
            \hline
              P&$P_1$&$P_2$&$P_3$&$P_4$&$P_5$&$P_6$&$P_7$&$P_8$&$P_9$&$P_{10}$&$P_{11}$&$P_{12}$&$P_{13}$&$P_{14}$&$P_{15}$&$P_{16}$\\
            \hline
          \end{tabular}
      }
    \end{table}
    \begin{itemize}
      \item [(6)]式(2.2)と(2.3),(2.5)と式(3.1)を用いて、表\ref{tab:実験1の結果}を作成した。
      \begin{equation}
        \begin{split}
            &Z=\dfrac{V}{I}
        \end{split}
      \end{equation}
    \end{itemize}

  \subsubsection{実験2}
    \begin{itemize}
      \item [(1)]実験1の実験を行った後に、電圧$V$、周波数$f$、インダクタンス$L$を計算できた1つの抵抗について、交流電源の電圧$V$、抵抗の端子間電圧$V_R$の波形をオシロスコープのCH1およびCH2で同時に観測した。この時両波形形の基準(GND)線を画面中央線に合わせた。
      \item [(2)]次に、オシロスコープのMEASURE機能とCURSOR機能を使ってCH1(電圧$V$)とCH2(抵抗の端子間電圧$V_R$)の最大値、波形の周期$T$、位相の時間差$t$をそれぞれをUSBに保存した。次に、CURSOR機能を使って波形の周期$T$と位相(CH1とCH2)の時間差$t$を表示させて、それぞれをUSBに保存した。
    \end{itemize}

\newpage
\subsection{実験機器}
  今回の実験で使用した機器を以下の表\ref{tab:使用機器}に示す。
  \begin{table}[H]
  \caption{使用機器}\label{tab:使用機器}
  \begin{center}
    \scalebox{1}{
      \begin{tabular}{|cccc|}
        \hline
          使用機器&規格&物品管理番号&製造会社\\
        \hline
        ディジタルストレージオシロスコープ&TBS1022&21/25&Tektronix\\
        ダイヤル式可変抵抗器&RD-56A&L06-000268&SHIMADZU\\
        関数発生器&AG-205&24/25&TEXIO\\
        \hline
      \end{tabular}
  }
  \end{center}
  \end{table}
  
\section{実験結果}
  今回の実験結果を、表4.1と表4.2、図4.1と4.2と4.3に示す。最初に、実験2で保存したオシロスコープの画面を図\ref{tab:osc_result}以下に示す。しかし、今回保存したものではCH1(インダクタンス$L$の電圧$V$)とCH2(抵抗$R$の電圧$V$)の値が同じなため、電圧降下より電流$I$がどちらかの素子において流れていないことになるので、実験2は失敗したと考えられる。
  \begin{figure}[H]
    \centering
      \includegraphics[keepaspectratio,scale=0.35]{figure/osc_result.pdf}
      \caption{抵抗の端子間電圧$V_R$と交流電源の電圧$V$、周期$T$と位相差$t$}\label{fig:osc_result}
  \end{figure}

  \newpage
  次に、実験1の実験結果である設定値の抵抗$R$、測定値抵抗の端子間電圧$V_R$、式(2.2)から求めた計算値の電流値$I$、式(4.1)からインピーダンス$Z$、式(2.5)から角度$\theta$を求めたものを、表\ref{tab:実験1の結果}に示す。また、ポイント$P_1$~$P_7$と$P_{15}$~$P_{16}$では、オシロスコープの画面においてCH2の測定結果が?が表示されていた。これは、オシロスコープ内では論理的に計算は行って表示しているが、1目盛りあたりの電圧値が小さいもしくは大きい等の理由からオシロスコープが正しく表示できていないことを示しているときに?と出る。このような理由から、?が表示されていたところは測定が行うことができなかったためこの表\ref{tab:実験1の結果}では「-」で示している。

  \begin{table}[H]
  \caption{実験1の結果(電圧$V=\SI{2.0}{\volt}$、周波数$f=\SI{100}{\kilo\hertz}$)}\label{tab:実験1の結果}
  \begin{center}
    \scalebox{1}{
      \begin{tabular}{|c|c|c|c|c|c|c|c|}
        \hline
        P&$R$ / \si{\ohm}&$V_R$ /V&$I$ / $\si{\milli\ampere}$& $Z$ / $\si{\kilo\ohm}$&$L$ / $\si{\milli\henry}$&$\theta$ / $^\circ$\\
        \hline
          $P_1$   &10&-&-&-&-&-\\
          $P_2$   &16&-&-&-&-&-\\
          $P_3$   &25&-&-&-&-&-\\
          $P_4$   &40&-&-&-&-&-\\
          $P_5$   &63&-&-&-&-&-\\
          $P_6$   &100&-&-&-&-&-\\
          $P_7$   &160&-&-&-&-&-\\
          $P_8$   &250&726m&2.90&0.690&1.02&68.72\\
          $P_9$   &400&1.04&2.60&0.770&1.05&58.67\\
          $P_{10}$   &630&1.39&2.20&0.910&1.05&45.97\\
          $P_{11}$   &1000&1.67&1.67&1.20&1.06&33.38\\
          $P_{12}$   &1600&1.83&1.14&1.75&1.13&23.79\\
          $P_{13}$   &2500&1.93&0.772&2.59&1.08&15.20\\
          $P_{14}$   &4000&1.99&0.500&4.02&0.637&5.7\\
          $P_{15}$      &6300&-&-&-&-&-\\
          $P_{16}$      &10000&-&-&-&-&-\\
        \hline
      \end{tabular}
  }
  \end{center}
  \end{table}

  次に、実験2の実験結果である表\ref{tab:実験2の結果}では、$V$の最大値と$V_R$の最大値と周期$T$と時間差$t$はそれぞれ図\ref{fig:osc_result}にあるオシロスコープの画面の結果である。また、電流は$I$と位相差$\theta$は指導書を参考に以下の式から求めることができる。

  \begin{equation}
    \begin{split}
        &I=\dfrac{\left(\dfrac{V_{Rmax}}{\sqrt{2}}\right)}{R}
    \end{split}
  \end{equation}
  \begin{equation}
    \begin{split}
        &\theta=360 \times \dfrac{t}{T}
    \end{split}
  \end{equation}
  \\
  \begin{table}[H]
  \caption{実験結果2の結果(電圧$V=\SI{2.0}{\volt}$、周波数$f=\SI{100}{\kilo\hertz}$、$C=\SI{0.11}{\micro\farad})$}\label{tab:実験2の結果}
  \begin{center}
    \scalebox{1}{
      \begin{tabular}{|c|c|c|c|c|c|c|}
        \hline
        抵抗&$V$の最大値&$V_R$の最大値&周期&時間差&電流&位相差\\
        $R$ / $\si{\kilo\ohm}$& $V_{max}$ / $V$ &$V_{Rmax}$ / $V$& $T$ / $\si{\micro\second}$ & $t$ / $\si{\micro\second}$ & $I$ / $\si{\milli\ampere}$ & $\theta$ / $^\circ$\\
        \hline
        \hline
        1&2.84&2.84&9.970&9.930&0.500&5.7\\
        \hline
      \end{tabular}
  }
  \end{center}
  \end{table}

  次に、図\ref{fig:実験結果のV_Rのベクトル軌跡}で表\ref{tab:実験1の結果}の抵抗の端子間電圧$V_R$の値と$\theta$の値を用いたベクトル軌跡を示す。
  \begin{figure}[H]
    \begin{center}
      \begin{circuitikz}
        \draw [->,>=stealth,very thick](0,0)--(10,0);
        \draw (5,1) node[anchor=north]{$\vec{V}$};
        \draw (10.5,1) node[anchor=north]{$\SI{2}{\volt}$};

        \draw [->,>=stealth,very thick,rotate around={-68.72:(0,0)}](0,0)--(0.726*5,0);
        \draw (,-3.5) node[anchor=north]{$P_8$};
        \draw (,-4)node[anchor=north] {$\vec{V}_{R8}$};

        \draw [->,>=stealth,very thick,rotate around={-58.67:(0,0)}](0,0)--(1.04*5,0);
        \draw (2.8,-4.6) node[anchor=north]{$P_9$};
        \draw (2.8,-5.1)node[anchor=north] {$\vec{V}_{R9}$};

        \draw [->,>=stealth,very thick,rotate around={-45.97:(0,0)}](0,0)--(1.39*5,0);
        \draw (5,-5.2) node[anchor=north]{$P_{10}$};
        \draw (5,-5.7)node[anchor=north] {$\vec{V}_{R10}$};

        \draw [->,>=stealth,very thick,rotate around={-33.38:(0,0)}](0,0)--(1.67*5,0);
        \draw (7,-4.6) node[anchor=north]{$P_{11}$};
        \draw (7,-5.1)node[anchor=north] {$\vec{V}_{R11}$};

        \draw [->,>=stealth,very thick,rotate around={-23.79:(0,0)}](0,0)--(1.83*5,0);
        \draw (8.9,-3.6) node[anchor=north]{$P_{12}$};
        \draw (9.6,-3.6)node[anchor=north] {$\vec{V}_{R12}$};

        \draw [->,>=stealth,very thick,rotate around={-15.20:(0,0)}](0,0)--(1.93*5,0);
        \draw (9.7,-2.3) node[anchor=north]{$P_{13}$};
        \draw (10.5,-2.3)node[anchor=north] {$\vec{V}_{R13}$};

        \draw [->,>=stealth,very thick,rotate around={-5.7:(0,0)}](0,0)--(1.99*5,0);
        \draw (10.4,-1) node[anchor=north]{$P_{14}$};
        \draw (11.3,-1)node[anchor=north] {$\vec{V}_{R14}$};

        \draw [->,>=stealth,semithick] (7,-6) arc (-70:-40:5);
        \draw (8.3,-6.2)node[anchor=north] {$R$増加};
        \draw [dashed,semithick](0,0) arc (-180:0:5);
      \end{circuitikz}
    \end{center}
    \caption{実験結果の$V_R$のベクトル軌跡}\label{fig:実験結果のV_Rのベクトル軌跡}
  \end{figure}
  \newpage
  次に、図\ref{fig:実験結果のIのベクトル軌跡}で表\ref{tab:実験1の結果}の抵抗の端子間電圧$I$の値と$\theta$の値を用いたベクトル軌跡を示す。
  \begin{figure}[H]
    \begin{center}
      \begin{circuitikz}
        \draw [->,>=stealth,very thick](0,0)--(2*2.5,0);
        \draw (2.5,)node[anchor=north] {$\vec{V}$};
        \draw (5,)node[anchor=north] {$\SI{2}{\volt}$};

        \draw [->,>=stealth,very thick](0,0)--(0,-3.12*2.5);
        \draw [<->,>=stealth,semithick] (-0.5,0)--(-0.5,-3.12*2.5);
        \draw (-1,-3.12*1.25)node[anchor=north] {$\dfrac{V}{\omega L}$};
        \draw (0,-3.12*2.6)node[anchor=north] {$\SI{3.12}{\milli\ampere}$};
        
        \draw [->,>=stealth,very thick,rotate around={-68.72:(0,0)}](0,0)--(2.90*2.5,0);
        \draw (2.5,-6.9)node[anchor=north] {$\vec{I}_{8}$};
        \draw (3,-6.8)node[anchor=north] {$P_8$};
        
        \draw [->,>=stealth,very thick,rotate around={-58.67:(0,0)}](0,0)--(2.60*2.5,0);
        \draw (3.5,-6)node[anchor=north] {$\vec{I}_{9}$};
        \draw (3.8,-5.6)node[anchor=north] {$P_9$};

        \draw [->,>=stealth,very thick,rotate around={-45.97:(0,0)}](0,0)--(2.20*2.5,0);
        \draw (3.5,-3.9)node[anchor=north] {$\vec{I}_{10}$};
        \draw (4.3,-3.9)node[anchor=north] {$P_{10}$};

        \draw [->,>=stealth,very thick,rotate around={-33.38:(0,0)}](0,0)--(1.67*2.5,0);
        \draw (3.3,-2.2)node[anchor=north] {$\vec{I}_{11}$};
        \draw (4,-2.2)node[anchor=north] {$P_{11}$};

        \draw [->,>=stealth,very thick,rotate around={-23.79:(0,0)}](0,0)--(1.14*2.5,0);
        \draw (3.8,-0.7)node[anchor=north] {$\vec{I}_{12}$};
        \draw (3.2,-0.8)node[anchor=north] {$P_{12}$};
        
        \draw [->,>=stealth,very thick,rotate around={-15.20:(0,0)}](0,0)--(0.772*2.5,0);
        \draw (3,-0.2)node[anchor=north] {$\vec{I}_{13}$};
        \draw (2.4,-0.2)node[anchor=north] {$P_{13}$};

        \draw [->,>=stealth,very thick,rotate around={-5.7:(0,0)}](0,0)--(0.5*2.5,0);
        \draw (1,0.7)node[anchor=north] {$\vec{I}_{14}$};
        \draw (1.5,0.6)node[anchor=north] {$P_{14}$};

        \draw [dashed,semithick](0,0) arc (90:-90:3.12*1.25);
      \end{circuitikz}
    \end{center}
    \caption{実験結果の$I$のベクトル軌跡}\label{fig:実験結果のIのベクトル軌跡}
  \end{figure}

\section{考察}
  \begin{itemize}
    \item [(1)]電圧、電流のベクトル軌跡は、図2.3(a)、図2.3(b)のように半円になるが、実際はどうなったか?描くのに苦労した点はどこか。\\電圧のベクトル軌跡は、原理のようにそれぞれのベクトル終点が弧を描くようなベクトルになっている。しかし、電流のベクトル軌跡では概ね原理と同じようなベクトル軌跡を描いているが半円の点線ぶぶんに届いていないところがあった。
    \item [(2)]抵抗$R$を$L$が計算できた範囲で変えたとき、インピーダンス$Z$のベクトル軌跡はどうなるか?\\
    \begin{figure}[H]
      \begin{center}
        \begin{circuitikz}
          \draw [->,>=stealth,thin](0,-3)--(0,3);
          \draw [->,>=stealth,thin](-1,0)--(5,0);

          \draw [->,>=stealth,semithick,rotate around={68.72:(0,0)}](0,0)--(0.69,0);
          \draw [->,>=stealth,semithick,rotate around={58.67:(0,0)}](0,0)--(0.77,0);
          \draw [->,>=stealth,semithick,rotate around={45.97:(0,0)}](0,0)--(0.91,0);
          \draw [->,>=stealth,semithick,rotate around={33.38:(0,0)}](0,0)--(1.2,0);
          \draw [->,>=stealth,semithick,rotate around={23.79:(0,0)}](0,0)--(1.75,0);
          \draw [->,>=stealth,semithick,rotate around={15.20:(0,0)}](0,0)--(2.59,0);
          \draw [->,>=stealth,semithick,rotate around={5.7:(0,0)}](0,0)--(4.02,0);

          \draw [->,>=stealth,very thick](0,0)--(4,0);
          \draw [->,>=stealth,very thick](0,0)--(0,0.4);

          \draw [dashed,thin](0,0.4)--(4,0.4);
          \draw [dashed,thin](4,0)--(4,0.4);

          \draw (0,4) node[anchor=north]{虚軸};
          \draw (0,3.5) node[anchor=north]{(Im)};
          \draw (6,0.25) node[anchor=north]{実軸};
          \draw (6,-0.25) node[anchor=north]{(Re)};

          \draw (2,-0.5) node[anchor=north]{$R$};
          \draw (-0.5,0.5) node[anchor=north]{$\omega L$};

          \scalebox{0.5}{
            \draw (0.4,1.9)node[anchor=north]{$P_8$};
            \draw (0.8,1.9)node[anchor=north]{$P_{9}$};
            \draw (1.3,1.9)node[anchor=north]{$P_{10}$};
            \draw (2.1,1.9)node[anchor=north]{$P_{11}$};
            \draw (3.3,1.9)node[anchor=north]{$P_{12}$};
            \draw (5,1.9)node[anchor=north]{$P_{13}$};
            \draw (8,1.4)node[anchor=north]{$P_{14}$};
          }

          \draw (-0.25,-0.05) node[anchor=north]{0};
        \end{circuitikz}
      \end{center}
      \caption{RL回路のインピーダンス$Z$の複素数表示 }\label{fig:RL_Z}
    \end{figure}
    \item [(3)]表4.1と表4.2の電流$I$、位相角$\theta$を比較、検討する。\\抵抗$R$の値が増えると、電流$I$が反比例して減っていることが読み取れる。これは、直列のRLでは流れる電流は等しく、またインダクタンス$R_L$は分数ではないため、抵抗が増えてインダクタンスも増えれば、オームの法則より反比例して増えるということから実験は成功したと考えられる。
  \end{itemize}

\section{まとめ}
  今回の実験を通して、電圧等がベクトルであることを理解することができたので、目的を達成することができたと思います。しかし、実験中にCH1とCH2の電圧値が同じになっているにもかかわらず、オシロスコープの画面を記録してしまいました。次回からは何を記録するのかを注意深く確認しながら実験を行っていくようにしたいと思います。

\end{document}